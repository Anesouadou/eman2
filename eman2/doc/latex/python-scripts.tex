\section{Writing Python Scripts}
\label{WRITING-PYTHON-SCRIPTS} \index{Python!Writing Scripts}

\subsection{Logging}
    EMAN2 includes built-in functionality to create a record of
    what commands are executed, using what parameters.  This
    information is stored in the file ``.eman2log'' located in the directory
    the commands are run.  

    In order to include records of new programs into the log, use the functions
    E2init([\textit{args}]) to initiate an entry and the function
    E2end() to finalize the entry.

    For example:\begin{verbatim}
logger = E2init (args)
.
. # interesting code here
.
E2end (logger)
\end{verbatim}

\subsection{Command-Line Argument Parsing}
    For convenience, the OptionParse module is generally used to handle
    command-line argument parsing.  If another parser is used, it is
    asked that the input argument format conform to the standards of
    OptionParser. Generally this means using a ``-'' in front of
    single letter options such as ``-v'' and using ``--'' in front of
    longer options names such as ``--verbose'' or ``--limit=15.4''

\subsection{Code Modularity}
    It is recommended that all program code be separated into
    1 or more functions rather directly written at the base level.  This allows
    programs to be run either individually or imported as modules into
    other programs that wish to use their functionality.

    Normally the base level of the program only includes
    \begin{verbatim}if __name__=="__main__":
    main()\end{verbatim}
    where main() is the function that actually begins completing the
    program's work.
    