\section {Image Header Attribute Naming Conventions}
\label{IMAGE-HEADER-ATTRIBUTES}
\index{Images!Attrib. Naming}

EMAN2 uses a unique name to reference each image header attribute. The
following table lists the conventions:

\begin{longtable}{|l|l|p{3.25in}|}

%\caption[Image Attribute Naming]{Image Attribute Namimg} \\

\hline 
\textbf{Attribute Name} &
\textbf{Data Type} & 
\textbf{Attribute Description} \\ \hline 
%\endfirsthead
\endhead

\hline \multicolumn{3}{r}{{Continued on next page}} \\
\endfoot
\hline \hline
\caption[Image Attribute Naming]{Image Attribute Namimg}
\endlastfoot

apix\_x &	float & 	pixel x size in � \\ \hline 
apix\_y &	float & 	pixel y size in � \\ \hline
apix\_z &	float &	pixel z size in � \\ \hline
nx &	int &	image x-direction size in number of pixels \\ \hline
ny & 	int &	image y-direction size in number of pixels \\ \hline
nz &	int & 	image z-direction size in number of pixels \\ \hline
origin\_row &	float & 	origin location along row (x) in
number of pixels \\ \hline
origin\_col &	float &	origin location along column (y) in number of
pixels \\ \hline
origin\_sec &	float &	origin location along section (z) in number of
pixels \\ \hline
minimum&
	float&	minimum pixel density value in the image \\ \hline
maximum&
	float&	maximum pixel density value in the image\\ \hline
mean&
	float&	mean pixel density value in the image\\ \hline
sigma&
	float&
	sigma  pixel density value in the image\\ \hline
square\_sum&
	float&	sum of the squares of each pixel density value in the image\\ \hline
mean\_nonzero&
	float&	mean pixel density value in the image, ignoring pixel whose value =0.\\ \hline
sigma\_nonzero&
	float&	sigma  pixel density value in the image, ignoring pixel whose value =0.\\ \hline
kurtosis&
	float&	kurtosis pixel density value in the image\\ \hline
skewness&
	float&	skewness pixel density value in the image\\ \hline
orentation\_convention&
	string&
	Euler orientation Convention, "EMAN", "MRC", etc\\ \hline
rot\_alt&
	float&
	EMAN convention: alt\\ \hline
rot\_az&
	float&
	EMAN convention: az\\ \hline
rot\_phi&
	float&
	EMAN convention: phi\\ \hline
datatype&	int&
	pixel data type in the image physical file (e.g., float, int, double, etc). This is an enumeration type.\\ \hline
is\_complex&	int&
	1 if this image is a complex image. 0 if it is a real image.\\ \hline
is\_ri&	int&
	1 if this is a complex image and is in real/imaginary format; 0 otherwise.\\ \hline
avgnimg&
	int&
	If the image is the result of averaging, avgnimg is the number of images used in averaging. Otherwise, avgnimg is 0.\\ \hline
label&
	string&
	user-defined label for this image. It is like 'name' in EMAN1.\\ \hline
image\_filename&
	string&
	the physical image filename where this image comes from.\\ \hline
image\_index&
	int&
	image number in file pointed to by path.\\ \hline
DM3.exposure\_number&
	int&
	DM3 image only. exposure number.\\ \hline
DM3.exposure\_time&
	double&
	DM3 image only. exposure time in seconds.\\ \hline
DM3.zoom&
	double&
	zoom\\ \hline
DM3.antiblooming&
	int&
	antiblooming\\ \hline
DM3.magnification&
	double&
	Indicated magnification\\ \hline
DM3.frame\_type&
	string&
	frame\_type\\ \hline
DM3.camera\_x&
	int& 	DM3 image only. camera size  along x direction in number of pixels.\\ \hline
DM3.camera\_y&
	int& 	DM3 image only. camera size along y direction in number of pixels.\\ \hline
DM3.binning\_x&
	int& 	DM3 image only. Binning \#0\\ \hline
DM3.binning\_y&
	int&
	DM3 image only. Binning \#1\\ \hline
MRC.nlabels&
	int&
	MRC image only.  Number of comment labels in a MRC image.\\ \hline
MRC.labelN&
	string&
	MRC image only. The Nth label, where N is an integer from 1 to 10.\\ \hline
LST.filenum&
	int&
	LIST image only, file number in FileSystem server\\ \hline
LST.reffile&
	string&
	LIST image only, reference file\\ \hline
LST.refn&
	int&
	LIST image only, reference image number\\ \hline
LST.comment&
	string&
	LIST image only,  comment\\ \hline
micrograph\_id&
	string&
	Micrograph ID.\\ \hline
particle\_center\_x&
	float&
	x coordinate of particle location in the micrograph\\ \hline
partcle\_center\_y&
	float&
	y coordinate of particle location in the micrograph\\ \hline
ctf\_dimension&
	int&
	CTF dimension, nD, where n is 1, 2, or 3.\\ \hline



\end{longtable}