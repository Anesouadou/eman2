\section{e2boxer.py} \index{Program Manuals!e2boxer.py} \label{e2boxer.py}
\subsection{Usage}
usage: e2boxer.py [options] $<$image$>$


\subsection{Description}
Automatic and manual particle selection. This version is specifically aimed at square boxes

for single particle analysis.




\subsection{Options}

\begin{longtable}{|c||p{3.5in}|}
\hline \bf{Argument} & \bf{Description}\endhead
\hline \multicolumn{2}{r}{{Continued on next page}} \endfoot
\hline \hline \caption[e2boxer.py Options]{e2boxer.py Options}\endlastfoot
\\\hline   --version  &  show program's version number and exit
\\\hline   -h, --help  &  show this help message and exit
\\\hline   --gui  &  Start the GUI for interactive boxing
\\\hline   -B  &  BOX, --box=BOX Box size in pixels
\\\hline   -P  &  PTCLSIZE, --ptclsize=PTCLSIZE Approximate size (diameter) of the particle in pixels. Not required if reference particles are provided.
\\\hline   -R  &  REFPTCL, --refptcl=REFPTCL A stack of reference images. Must have the same scale as the image being boxed.
\\\hline   -V  &  REFVOL, --refvol=REFVOL A 3D model to use as a reference for autoboxing
\\\hline   -S  &  SYM, --sym=SYM Symmetry of the 3D model
\\\hline   -A  &  AUTO, --auto=AUTO Autobox using specified method: circle, ref
\\\hline   -T  &  THRESHOLD, --threshold=THRESHOLD Threshold for keeping particles. 0-4, 0 excludes all, 4 keeps all.
\\\hline   --nretest=NRETEST  &  Number of reference images (starting with the first) to use in the final test for particle quality.
\\\hline   --retestlist=RETESTLIST  &  Comma separated list of image numbers for retest cycle
\\\hline   --norm  &  Edgenormalize boxed particles
\\\hline   --savealiref  &  Stores intermediate aligned particle images in boxali.hdf. Mainly for debugging.
\\\hline
 --farfocus=FARFOCUS  &  filename or 'next', name of an aligned far from focus image for preliminary boxing
\\\hline
\end{longtable}
