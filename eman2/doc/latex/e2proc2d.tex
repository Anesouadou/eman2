\section{e2proc2d.py} \index{Program Manuals!e2proc2d.py} \label{e2proc2d.py}
\subsection{Usage}
usage: e2proc2d.py options inputfile outputfile


\subsection{Description}

\subsection{Options}

\begin{longtable}{|c||p{3.5in}|}
\hline \bf{Argument} & \bf{Description}\endhead
\hline \multicolumn{2}{r}{{Continued on next page}} \endfoot
\hline \hline \caption[e2proc2d.py Options]{e2proc2d.py Options}\endlastfoot
\\\hline   --version  &  show program's version number and exit
\\\hline   -h, --help  &  show this help message and exit
\\\hline   --apix=APIX  &  A/pixel for S scaling
\\\hline   --average  &  Averages all input images (without alignment) and writes a single (normalized) output image
\\\hline   --calcsf=n  &  outputfile calculate a radial structure factor for the image and write it to the output file, must specify apix. divide into $<$n$>$ angular bins
\\\hline   --clip=xsize  &  ysize Define the output image size. CLIP=xsize ysize
\\\hline   --ctfsplit  &  Splits the input file into output files with the same CTF parameters
\\\hline   --exclude=exclude-list-file  &  Excludes image numbers in EXCLUDE file
\\\hline   --fftavg=filename  &  Incoherent Fourier average of all images and write a single power spectrum image
\\\hline   --filter=filtername:param1=val1:param2=val2  &  apply a filter named 'filtername' with all its parameters/values.
\\\hline   --first=n  &  the first image in the input to process [0 - n-1])
\\\hline   --inplace  &  Output overwrites input, USE SAME FILENAME, DO NOT 'clip' images.
\\\hline   --interlv=interleave-file  &  Specifies a 2nd input file. Output will be 2 files interleaved.
\\\hline   --last=n  &  the last image in the input to process
\\\hline   --list=listfile  &  Works only on the image numbers in LIST file
\\\hline   --meanshrink=n  &  Reduce an image size by an integral scaling factor using average. Clip is not required.
\\\hline   --mraprep  &  this is an experimental option
\\\hline   --norefs  &  Skip any input images which are marked as references (usually used with classes.*)
\\\hline   --outtype=image-type  &  output image format, mrc, imagic, hdf, etc
\\\hline   --plt=plt-file  &  output the orientations in IMAGIC .plt file format
\\\hline   --radon  &  Do Radon transform
\\\hline   --rfp  &  this is an experimental option
\\\hline   --scale=f  &  Scale by specified scaling factor. Clip must also be specified to change the dimensions of the output map.
\\\hline   --selfcl=steps  &  mode Output file will be a 180x180 self-common lines map for each image.
\\\hline   --setsfpairs  &  Applies the radial structure factor of the 1st image to the 2nd, the 3rd to the 4th, etc
\\\hline   --shrink=n  &  Reduce an image size by an integral scaling factor, uses median filter. Clip is not required.
\\\hline   --split=n  &  Splits the input file into a set of n output files
\\\hline
 --verbose=n  &  verbose level [1-4]
\\\hline
\end{longtable}
