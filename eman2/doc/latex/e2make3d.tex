\section{e2make3d.py} \index{Program Manuals!e2make3d.py} \label{e2make3d.py}
\subsection{Usage}
usage: e2make3d.py $<$input file$>$ [options]


\subsection{Description}

\subsection{Options}

\begin{longtable}{|c||p{3.5in}|}
\hline \bf{Argument} & \bf{Description}\endhead
\hline \multicolumn{2}{r}{{Continued on next page}} \endfoot
\hline \hline \caption[e2make3d.py Options]{e2make3d.py Options}\endlastfoot
\\\hline   --version  &  show program's version number and exit
\\\hline   -h, --help  &  show this help message and exit
\\\hline   --out=FILENAME  &  Output 3D MRC file
\\\hline   --lowmem  &  Read in images only when needed instead of preloading all in memory
\\\hline   --sym=SYM  &  Set the symmetry; if no value is given then the model is assumed to have no symmetry. Choices are: i, c, d, t, icos, or oct
\\\hline   --pad=PAD  &  To reduce Fourier artifacts, the model is typically padded by ~25\%
\\\hline   --mode=MODE  &  Specifies the interpolation size (1 to 6), 2 is the default
\\\hline   --recon=RECON\_TYPE  &  Reconstructor to use
\\\hline   --quiet  &  Quiet output
\\\hline   --snrfile=SNRFILE  &  Use a SNR file (as generated by classalignall) for proper 3D Wiener filtration
\\\hline   --goodbad  &  Saves the used and unused class averages in 2 files
\\\hline   --log  &  Averages the log of the projections
\\\hline   --inorm  &  This will use a special weighting scheme to compensate for poor SNR sampling on the unit sphere
\\\hline   --fftmerge=FFTMERGE  &  Fourier model to merge with real model
\\\hline   --savenorm  &  Saves the normalization map to norm.mrc
\\\hline   --hard=HARD  &  This specifies how well the class averages must match the model to be included, 25 is typical
\\\hline   --noweight  &  Normally the class averages are weighted by the number of raw particles used, this disables that
\\\hline   --mask=MASK  &  Real-space mask radius
\\\hline   --apix=APIX  &  Set the sampling (angstrom/pixel)
\\\hline   --keep=SIGM  &  An alternative to 'hard'
\\\hline
 --resmap=RESMAP  &  Generates a 'resolution map' in another 3D MRC file
\\\hline
\end{longtable}
