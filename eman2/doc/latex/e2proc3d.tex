\section{e2proc3d.py} \index{Program Manuals!e2proc3d.py} \label{e2proc3d.py}
\subsection{Usage}
usage: e2proc3d.py options inputfile outputfile


\subsection{Description}

\subsection{Options}

\begin{longtable}{|c||p{3.5in}|}
\hline \bf{Argument} & \bf{Description}\endhead
\hline \multicolumn{2}{r}{{Continued on next page}} \endfoot
\hline \hline \caption[e2proc3d.py Options]{e2proc3d.py Options}\endlastfoot
\\\hline   -h, --help  &  show this help message and exit
\\\hline   --shrink=n  &  Shrinks the image by integer n using median filter
\\\hline   --scale=n  &  Rescales the image by 'n', generally used with clip option.
\\\hline   --clip=x  &  y z xc yc zc Make the output have this size, no scaling.
\\\hline   --fftclip=x  &  y z Make the output have this size, rescaling by padding FFT.
\\\hline   --filter=filtername:param1=val1:param2=val2  &  Apply a filter named 'filtername' with all its parameters/values.
\\\hline   --apix=APIX  &  A/pixel for S scaling
\\\hline   --origin=x  &  y z Set the coordinates for the pixel (0,0,0).
\\\hline   --mult=f  &  Scales the densities by 'f' in the output
\\\hline   --add=f  &  Adds a constant 'f' to the densities
\\\hline   --calcsf=outputfile  &  Calculate a radial structure factor. Must specify apix.
\\\hline   --tophalf  &  The output only keeps the top half map
\\\hline   --icos5fhalfmap  &  The input is the icos 5f top half map generated by the 'tophalf' option
\\\hline
 --outtype=image-type  &  Set output image format, mrc, imagic, hdf, etc
\\\hline
\end{longtable}
