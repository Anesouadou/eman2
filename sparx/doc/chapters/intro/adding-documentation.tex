\section{Adding Documentation}

This documentation for SPARX is made using \LaTeX ~(see the
\href{http://www.latex-project.org/}{\LaTeX ~homepage} for more
information).  There are two files, {\it sparx.tex} and {\it TOC.tex},
which respectively define the basic the layout of the entire document and
the order in which sections of the document are included.

When adding new documentation to this existing document, no preamble
tags should be included in the new sections.  Preamble information can
removed either manually or using the python script rm\_preamble.py 
(see \ref{ADDING-DOC-RMPRE} for more information).  

The recommended steps for adding new documentation go as follows:
\begin{enumerate}
  \item Create a new stand alone \LaTeX document
  \item Remove the preamble information from the new file
  (\ref{ADDING-DOC-RMPRE})
  \item Edit TOC.tex (\ref{ADDING-DOC-TOC})
  \item Run ``make'' to create the new documentation file
\end{enumerate}


\subsection{Removing Preambles} \label{ADDING-DOC-RMPRE}

The preamble of a \LaTeX ~document is used to specify various global
parameters of the document.  In a stand alone document, all commands
coming before and including {$\backslash$begin\{document\}} are part of the
preamble.  There is also a {$\backslash$end\{document\}} at the very end
of the document that should be removed.

Use the ``\%'' character at the  start of a line to comment it out.

The python script rm\_preamble.py can usually be used to automatically remove
the preamble and other tags that should not be used.
\begin {itemize}
  { 
  \item[\%] python rm\_preamble.py \textit{file1 file2 ...}}
\end{itemize}
The output will be placed in the \textit{file1}.tex,
\textit{file2}.tex, ...

\subsection{Editing TOC.tex} \label{ADDING-DOC-TOC}

TOC.tex contains the table of contents information for this document.
Add ``{$\backslash$input\{\textit{new file name}\}}'' in the appropriate
chapter to insert a new documentation file.

A new chapter can be added using the
``{$\backslash$chapter\{\textit{new chapter name}\}}'' command.
