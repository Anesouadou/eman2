%\documentclass{manual}
%\usepackage{makeidx}
%\begin{document}

\section{Goals}
  \label{GOALS}

\begin{enumerate}
  \item 
    EMAN2 should have a generic way to read/write various electronic
    microscopy image formats.
    \begin{itemize}
      \item
	Support all electron-microscopy image formats supported in EMAN
      \item
	Read an arbitrary convex 2D/3D region of any image format
      \item
	Separate reading/writing of image headers from image data
      \item
	Support averaging and shrinking when reading images
    \end{itemize}

    \item
      EMAN2 should make it easy to use and develop image processors.
      \begin{itemize}
	\item
	  It should be simple and easy to define a new filter in
	  either C++ or Python.
	\item
	  Allow users to define new filters that accept arbitrary numbers
	  of parameters in an arbitrary order
	\item
	  Utilize existing filters defined in EMAN1 system.
	\item
	  Filters should be able to be pipelined.
	\item
	  Each filter is identified with a meaningful name
      \end{itemize}

    \item
      The same design goals for filters should also apply to EM image
      aligners, image comparators, image averagers, image 3D
      reconstructors.

    \item
      EMAN2 should have a generic way to handle image translation,
      rotation, Euler angles. It should support basic geometry
      operations like vector and matrix.

    \item
      EMAN2 should allow different definitions of CTFs.

    \item
      EMAN2 should have a modular design to support different FFTW
      library at compile time.

    \item
      EMAN2 should support data processing on an arbitrary convex
      region of an image.
 
    \item
      EMAN2 should support a logging mechanism similar to that in EMAN1.

    \item
      EMAN2 should be designed for easy regression tests.

    \item
      EMAN2 should be fully documented using Doxygen style.

    \item
      EMAN2 should support elegant integration with Phoenix software:

      \begin{itemize}
	\item
	  Support data interchange with Phoenix CCTBX.
	\item
	  Implement basic image reconstruction tasks used in Phenix
	  GUI environment. 
      \end{itemize}
 
\end{enumerate}

%\end{document}