%\documentclass{manual}
%\usepackage{makeidx, hyperref}
%\begin{document}


\section{How to Install Boost Python}
\label{HELP-OPT-PROG-INST}
\index{Boost Python!Installation}
  \begin{enumerate}
    \item
      Download 'bjam' for your platform.
    \item
      Download boost source from http://www.boost.org. Assume the
      version is boost\_1\_32\_0.
      \color[named]{Blue}
      \begin{itemize} 
        \item[\%] cd /usr/local/src
	\item[\%] tar zxf boost\_1\_32\_0.tar.gz
	\item[\%] cd boost\_1\_32\_0.
      \end{itemize}
      \normalcolor

    \item
      Set up environment variables ''PYTHON\_ROOT'' and ''PYTHON\_VERSION''.
       For example, if your python is at /usr/bin/python then PYTHON\_ROOT is
       ''/usr''. If your python version is 2.2.X, then PYTHON\_VERSION
       is '2.2'.
       \begin{enumerate}
       \item 
	 check your shell: {\color[named]{Blue}\% echo \$SHELL}
       
       \item
	 if you are using bash/zsh, do
	 \color[named]{Blue}
	 \begin{itemize}
           \item[\%] export PYTHON\_VERSION=2.2
           \item[\%] export PYTHON\_ROOT=/usr
	 \end{itemize}
	 \normalcolor

          if you are using csh/tcsh, do
	  \color[named]{Blue}
	  \begin{itemize}
            \item[\%] setenv PYTHON\_VERSION 2.2
            \item[\%] setenv PYTHON\_ROOT /usr
	  \end{itemize}
	  \normalcolor
       \end{enumerate}

     \item
	  {\color[named]{Blue} cd libs/python/build}
    
     \item
       run 'bjam' with your options:
       \begin{itemize}
	 \item[-] linux-x86: {\color[named]{Blue}\% bjam}
	 \item[-] SGI Irix: {\color[named]{Blue}\% bjam ''-sTOOLS=mipspro''}
       \end{itemize}

     \item
       login as root
    
     \item
       \color[named]{Blue}
       \begin{itemize}
	 \item[\%] cp -df bin-stage/libboost\_python.so* /usr/local/lib
	 \item[\%] cd ../../..; cp -rf boost /usr/local/include
       \end{itemize}
       \normalcolor

  \end{enumerate}


\section{How to use your own version of python}

   If the python you want to use in your computer is not found by CMake,
    you may set up environment variables ''PYTHON\_ROOT'' and
    ''PYTHON\_VERSION''. For example, if your python is at
    /usr/local/python2.4/bin/python. PYTHON\_ROOT is
    ''/usr/local/python2.4''. if your python is 2.4.X, PYTHON\_VERSION
    is '2.4'.

\section{How to Install Numeric Python}
\index{Numeric Python!Installation}
   From the website
    \href{http://sourceforge.net/projects/numpy}{http://sourceforge.net/projects/numpy}
    and download 
    version 22.0 of Numeric Python.
 
    For windows, run the binary installer and the installation is
    complete.  Other users must download the source code and install
    manually as follows:

    \begin{itemize}
      \item[-] get source code Numeric-XX.Y.tar.gz
      \color[named]{Blue}
      \item[-] \% gunzip Numeric-XX.Y.tar.gz \\
	\% tar xf Numeric-XX.Y.tar
      \normalcolor
      \item[-] login as root
      \color[named]{Blue}
      \item[-] \% cd Numeric-XX.Y; \\
	\% python setup.py install
      \normalcolor
    \end{itemize}



%\end{document}