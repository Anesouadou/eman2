\section{Using the Log Class} \label{USING-LOGGER}
EMAN2 includes a built in class to handle writing log messages.  It is
highly recommended that all textual output be handled by this class.  

\subsection{Typical Usage}
It is sometimes necessary to specify what output level of the message
should be.
There are 4 log levels to choose from
\begin{itemize}
  \item \textit{ERROR\_LOG} - used for error messages
  \item \textit{WARNING\_LOG} - used for warning messages
  \item \textit{DEBUG\_LOG} - used for debugging messages
  \item \textit{VARIABLE\_LOG} - highly detailed debugging messages
\end{itemize}
The default value used is \textit{ERROR\_LOG}; to change the level use:
{\color[named]{BrickRed}\begin{verbatim} Log::logger()->set_level(Log::NEW_LOG_LEVEL); \end{verbatim}}
Next, output the log message using:
{\color[named]{BrickRed}\begin{verbatim} Log::logger()->OUTPUT_FUNC;\end{verbatim}}
where OUTPUT\_FUNC is one of the following:
\begin{itemize}
  \item \textit{\textbf{error}(string)} - log an error message
  \item \textit{\textbf{warn}(string)} - log a warning message
  \item \textit{\textbf{debug}(string)} - log a debugging message
  \item \textit{\textbf{variable}(string)} - log a very detailed
  debugging message
\end{itemize}
