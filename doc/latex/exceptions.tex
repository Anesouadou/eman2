%\documentclass[10pt]{manual}
%\usepackage{fullpage, graphicx, url}
%\setlength{\parskip}{1ex}
%\setlength{\parindent}{0ex}
%\title{EMAN2 Exception FAQ}
%\begin{document}

\section{Exceptions}

\label{EXCEPTIONS} 
\index{Exceptions}

\subsection{Using Exceptions}
\index{Exceptions!Usage}

\begin{itemize}
\item  Here is an example on throwing an exception:
%{\color[named]{BrickRed}
  \begin{verbatim}vector <float>EMData::calc_fourier_shell_correlation(EMData * with){
��� if (!with) {
��� ��� throw NullPointerException("NULL input image");
��� }

��� if (!EMUtil::is_same_size(this, with)) {
��� ��� LOGERR("images not same size");
��� ��� throw ImageFormatException( "images not same size");
��� }� 
//...
}\end{verbatim}%}

\item Here is an example on catching all possible exception
%{\color[named]{BrickRed}
  \begin{verbatim}void foo()
{
    EMData* e1 = new EMData();
    EMData* e2 = new EMData();
    try {
        e1->read_image("test1.mrc");
        e2->read_image("test2.mrc");
        vector<float> v = e1->calc_fourier_shell_correlation(e2);
    }
    catch (E2Exception & exception) {
        printf("%s\n", exception.what());
    }
}\end{verbatim}%}

\item Here is an example on catching a specific exception:
%{\color[named]{BrickRed}
\begin{verbatim}void foo()
{
    EMData* e1 = new EMData();
    EMData* e2 = new EMData();
    try {
        e1->read_image("test1.mrc");
        e2->read_image("test2.mrc");
        vector<float> v = e1->calc_fourier_shell_correlation(e2);
    }
    catch (_NullPointerException & exception) {
        printf("%s\n", exception.what());
    }
}\end{verbatim}%}


 
 
Note the ``\_'' before \_NullPointerException.


\end{itemize}

\subsection{Existing Exception Types}
\begin{itemize}
  \item
    See the complete EMAN2 Reference Guide
\end{itemize}

\subsection{Defining New Exception Classes}
\index{Exceptions!Adding New}
\begin{itemize}
  \item
    A XYZ Exception class is defined in the following way:
    \begin{itemize}
     \item
     It will extend E2Exception class.
     \item
     The class is named \_XYZException.
     \item
     The class has a function to return its name "XYZException".
     \item
     A macro called "XYZException" is defined to simplify the usage of
    \_XYZException class.
    \end{itemize}
  \item
     Here is the code for NullPointerException.
%     {\color[named]{BrickRed}
     \begin{verbatim}class _NullPointerException : public E2Exception {
  public:
        _NullPointerException(const string& file = "unknown",
                              int line = 0, 
                              const string& desc_str = "")
            : E2Exception(file, line, desc_str) {}
       
        const char *name() const{ return "NullPointerException"; }
       
  };

#define NullPointerException(desc) _NullPointerException(__FILE__,\
__LINE__,desc)\end{verbatim}% }

\end{itemize}



    


%\end{document}
