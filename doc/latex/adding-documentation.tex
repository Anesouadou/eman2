\section{Adding Documentation}

The majority of the documentation for EMAN2 is made using \LaTeX ~(see the
\href{http://www.latex-project.org/}{\LaTeX ~homepage} for more
information).  No preamble should be included in the new
documentation; preamble information can removed either manually or
using the python script rm\_preamble.py (see \ref{ADDING-DOC-RMPRE}
for more information).  

Documentation can be created using a WYSIWYG \LaTeX ~ editor such as
\href{http://www.lyx.org/}{LyX} or
\href{http://www.www.texmacs.org/}{TeXmacs}.  When using these
utilities do not create title pages, abstract sections, or tables of
content as this will cause integration problems with the rest of
documentation.

The recommended steps for adding new documentation go as follows:
\begin{enumerate}
  \item Create a new stand alone document (see \ref{ADDING-DOC-STANDS}
  for standards information)
  \item Remove the preamble information from the new file
  (\ref{ADDING-DOC-RMPRE})
  \item Edit TOC.tex (\ref{ADDING-DOC-TOC})
\end{enumerate}

\subsection{Document Standards} \label{ADDING-DOC-STANDS}
Please do not include the following sections/commands in new documentation:
\begin{itemize}
  \item
    $\backslash$title, $\backslash$maketitle
  \item
    $\backslash$tableofcontents
\end{itemize}
Edit the existing title and table of content information instead.


\subsection{Removing Preambles} \label{ADDING-DOC-RMPRE}

The preamble of a \LaTeX ~document is used to specify various global
parameters of the document.  In a stand alone document, all commands
coming before and including $\backslash$begin\{document\} are part of the
preamble.  There is also a $\backslash$end\{document\} at the very end
of the document that should be removed.

Use the ``\%'' character at the  start of a line to comment it out.

The python script rm\_preamble.py can usually be used to automatically remove
the preamble and other tags that should not be used.
\begin {itemize}
  \item[\%] python rm\_preamble.py \textit{file1 file2 ...}
\end{itemize}
The output will be placed in the \textit{file1}.tex,
\textit{file2}.tex, ...

\subsection{Editing TOC.tex} \label{ADDING-DOC-TOC}

TOC.tex contains the table of contents information for this document.
Add ``$\backslash$input\{\textit{new file name}\}'' in the appropriate
chapter to insert a new documentation file.
Of course a new chapter can always be added using the
``$\backslash$chapter\{\textit{new chapter name}\}'' command.


\subsection{Adding \LaTeX ~Packages}

The package inclusion tag for the documentation is located in
EMAN2.tex.  Add the package that is needed to the
$\backslash$usepackage command.
