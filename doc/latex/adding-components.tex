\section{Adding Components}
  \label{ADDING-COMPONENTS}
  \index{Adding Components}

There are two general procedures for adding new components to EMAN2.
The first option involves editing templates in order to integrate the
new component into the core while the second option involves directly
altering the existing core files.  

It is generally recommended that components should be added first
using templates and later directly added to existing core if 
desired.  When testing or refining a new components build times are much
faster if templates are used.  
\begin{itemize}
  \item[-]
    General Overview for Using Templates:
    \begin{enumerate}
    \item
      Edit the \textbf{template.h} and
      \textbf{template.cpp} files for the type of 
      component that is being added
    \item
      From EMAN2/src/build run:\\
      { \% make}
    \end{enumerate}
\end{itemize}

\subsection{Processors}
  \label{ADDING-PROCESSORS} 
  \index{Processors!Adding} 
  \index{Adding Components!Processors}

      \subsubsection{Using Templates:}
      Located in the plugin directory of the EMAN2 source
      (i.e. EMAN2/src/eman2/plugins) are various template files including
      \textbf{processor\_template.h} and \textbf{processor\_template.cpp}.  These are the files
      that will be used for new processor installation.  Begin by editing
      \textbf{processor\_template.h}. 

      \begin{enumerate}
	\item
	  Change the occurrences of "XYZ" in "XYZProcessor" with the name of
	  the new processor
	  \begin{itemize}
	    \item
	      Don't forget to change the string in get\_name() to the name of
	      the processor (this is the name that will be used to
	      call the processor)
	  \end{itemize}
	\item
	  Edit the string in get\_desc() with a brief description of the
	  processor.  Place a more detailed descriptions elsewhere such as
          before the class or before the functions (see
	  \ref{CODING-STYLE} for coding style information)
	\item
	  Define the processor's parameters in get\_param\_types()
	  \begin{itemize}
	    \item
	      A description string can be added as a third param to the
	      TypeDict::put() function to describe the variables 
	  \end{itemize}
	\item
	  In the class constructor of FilterFactorExt uncomment the line
	  "Factory $<$Processor$>$ ::add(\&dProcessor::NEW);"
      \end{enumerate}

      Now edit \textbf{processor\_template.cpp}
      
      \begin{itemize}
	\item
	  In processor() add the implementation code of the new
	  processor.
	  \begin{itemize}
	    \item
	      The existing template has sample code showing how to access the
	      variables that where defined in get\_para\_types()
	    \item
	      Note that the sample code included in the template is enclosed in
	      a conditional statement that essentially causes all of
	      the code tobe skipped.
	  \end{itemize}
      \end{itemize}

      Finally rebuild EMAN:
%      \color[named]{Blue}
      \begin{itemize}
	\item[\%] cd ../../build
	\item[\%] make
      \end{itemize}
      \normalcolor
      The new processor should now be available using the name that was
      specified in get\_name().



  \subsubsection{Adding Directly to the Core:}

  If the new processor code has already been created using the supplied
  templates, then adding to the core can be done as follows:
  \begin{enumerate}
    \item
      Open processor.h in src/eman2/libEM
    \item
      Copy the class you defined in processor\_template.h and paste it
      in the file
    \item
      Open processor.cpp (also located in src/eman2/libEM)
    \item
      Copy and paste the class from processor\_template.cpp to
      processor.cpp.  Header file include statements may also need to
      be copied.
    \item \label{REPEAT-STEP1}
      In the template class Factory located in the begining of
      processor.cpp add a line "force\_add(\&newProcessor::NEW)" where
      "newProcessor" is the name of the processor class that is being
      added 
   \item \label {REPEAT-STEP2}
     Rebuild EMAN2
  \end{enumerate}

  The instructions for adding a new processor without first using
  templates go as follows:
  \begin{enumerate}
    \item
      In src/eman2/libEM open processor.h
    \item
      Towards the end of the file there is an example class called
      XYZProcessor.  Follow the first 3 steps listed in "Using
      Templates". 
    \item
      Open processor.cpp and write an implementation for the process() 
      function that was just defined in processor.h
    \item
      Repeat the last two steps (\ref{REPEAT-STEP1} and \ref{REPEAT-STEP2})
      from the template installation
  \end{enumerate}

\subsection{Aligners} 
  \label{ADDING-ALIGNERS} 
  \index{Aligners!Adding New} 
  \index{Adding Components!Aligners}

  The instructions are the same as those for adding a new processor
  (\ref{ADDING-PROCESSORS}) except the files of interest are
  aligner\_template.h and aligner\_template.cpp.

  Also, when adding to the core, aligner.h and aligner.cpp should be altered.

\subsection{Averagers}
  \label{ADDING-AVERAGERS}
  \index{Averagers!Adding New} 
  \index{Adding Components!Averagers}

  The instructions are the same as those for adding a new processor
  (\ref{ADDING-PROCESSORS}) except the files of interest are
  averager\_template.h and averager\_template.cpp.

  Also, when adding to the core, averager.h and averager.cpp should be altered.

\subsection{Image Comparator}
  \label{ADDING-CMP} 
  \index{CMP!Adding New} 
  \index{Adding Components!CMP}

  The instructions are the same as those for adding a new processor
  (\ref{ADDING-PROCESSORS}) except the files of interest are
  cmp\_template.h and cmp\_template.cpp.

  Also, when adding to the core, cmp.h and cmp.cpp should be altered.


\subsection{IO Functionality}
  \label{ADDING-IO} 
  \index{IO!Adding New} 
  \index{Adding Components!IO}

  The instructions are the same as those for adding a new processor
  (\ref{ADDING-PROCESSORS}) except the files of interest are
  io\_template.h and io\_template.cpp.

  Also, when adding to the core, *****NEED TO CHECK THIS***** should be altered.


\subsection{Reconstructors}
  \label{ADDING-RECONSTRUCTORS} 
  \index{Reconstructors!Adding} 
  \index{Adding Components!Reconstructors}
  
  The instructions are the same as those for adding a new processor
  (\ref{ADDING-PROCESSORS}) except the files of interest are
  reconstructor\_template.h and reconstructor\_template.cpp.

  Also, when adding to the core, reconstructor.h and reconstructor.cpp should be altered.
